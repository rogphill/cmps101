\documentclass[a4paper,12pt]{article}
\usepackage{amssymb}

\title{Homework One} 
\author{Robert George Phillips (rogphill@ucsc.edu)}
\date{January 21, 2018}

\begin{document}
\maketitle

\section{}

\textbf{Question:} A common algorithm for sorting is Bubble-Sort. Consider an input array $A = [A[1], A[2], . . . , A[n]]$, with $n$ elements. You repeat the
following process $n$ times: for every $i$ from $1$ to $ n - 1$, if $A[i]$ and $A[i + 1]$ are out of order, you swap them. Write this out as pseudocode. Do a time complexity analysis of Bubble-Sort. Prove that the time complexity is $\Theta(n^2)$. Prove that the algorithm works (i.e. that the final array is sorted) by using loop invariants.\\


\subsection{Pseudocode}

\begin{verbatim}
A = {A[1], A[2], ..., A[n]}
i = 1
j = 1
swapper = 0

for i to n-1 {
    for j to n-1 {
        if A[j] > A[j+1] {
            swapper = A[j+1]
            A[j+1] = A[j]
            A[j]= swapper			
        }
    }
}
\end{verbatim}

\subsection{Time Complexity Analysis}

Bubble sort will run $n$ times in the inner loop, such that $1 \leq j \leq n-1$ for each time the outer loop iterates. Thus, bubble sort is $O(n*n=n^2)$.\\

$F(n) \in O(n^2)$\\

$G(n) \in O(n^2)$\\

$\lim_{n\to\infty} \frac{F(n)}{G(n)} = \frac{n^2}{n^2} = 1$\\

Since $1$ is a constant, $F(n) \in O(n^2)$ and $F(n) \in \Omega(n^2)$.\\

Thus,  $F(n) \in \Theta(n^2)$.\\

\subsection{Proof by Induction}

\textbf{Invariant:} After $k$ iterations of the outer loop, the last $k$ elements are the largest. Furthermore, the last $k$ elements are a permutation of the original array $A$ and are sorted in ascending order.\\

\textbf{Base case ($k=1$):} Vacuously true. After one iteration of the outer loop, the last element in array $A$ is the largest element in array $A$.\\

\textbf{Induction:} Suppose the invariant is true for $k$. Let us prove the invariant for $k+1$:\\

By induction, at the end of $k$ outer loops, everything in $A[(n-k+1), \ldots, n]$ are the largest $k$ elements in array $A$, and are sorted in ascending order.\\

When the $(k+1)$\textsuperscript{st} iteration of the outer loop ends, the last $(k+1)$ elements are the largest elements in array $A$, in sorted order, such that $A[(n-k)] \leq A[(n-k+1), \ldots, n]$.\\

Therefore, $A[(n-k+1), \ldots, n]$ is sorted.\\

$$\blacksquare$$\\

\section{}

\textbf{Question:} Use induction to prove the following statement: The number of subsets of $\{1, 2, \ldots, n\}$ having an odd number of elements is $2^{n-1}$. Clearly state your hypothesis, base case and inductive step.\\

\subsection{Hypothesis}

 There are $2^{n-1}$ subsets from the set $\{1, 2, \ldots, n\}$ that have an odd number of elements.\\

\subsection{Proof by Induction} 

\textbf{Base case ($n=1$)}: Vacuously true. $2^{n-1} = 2^{1-1} = 2^0 = 1$. There is only one subset that is odd, $\{1\}$.\\

\textbf{Induction:} Suppose a set of $n$ elements has $2^{n-1}$ subsets with an odd number of elements. Let us prove that a set of $n+1$ elements has $2^{n-1}$ subsets with an odd number of elements:\\

$2^{(n+1)-1}=2^{n}$\\

%Seen on Piazza, credit to colleague Mamon Alsalihy:
$2^{n-1}\cup (n+1) = 2^n$\\

Since we have a set of size $n$, the total number of subsets is $2^n$. Thus, there must be an equal number of even subsets and odd subsets such that $2^{n-1}+2^{n-1}=2^n$.\\

Therefore, the set $\{1, 2, \ldots, n, n+1\}$ has $2^{n-1}$ subsets with an odd number of elements and thus all sets have $2^{n-1}$ subsets with an odd number of elements.

$$\blacksquare$$\\

\section{}

\textbf{Question:}  Let $f(n) = a_{0} + a_{1} n + a_{2} n^{2} +
  \ldots + a_{k} n^{k}$ be a degree-$k$ polynomial, where every $a_{i} >
  0$. Show that $f(n) \in \Theta(n^{k})$. 
  Furthermore, show that $f(n) \notin O(n^{k\prime})$, for all $k\prime < k$.\\

\subsection{Asymptotic Analysis}

Let us take the highest degree polynomial in $F(n) = a_0+a_1n+a_2n^2+\ldots + a_kn^k$ where $a_i > 0$, and classify it as $O(n^k)$. Thus,\\

$F(n) = a_0+a_1n+a_2n^2+\ldots + a_kn^k = O(n^k)$\\

$G(n) = n^k = O(n^k)$\\

$\lim_{n\to\infty} \frac{F(n)}{G(n)} = \frac{a_0+a_1n+a_2n^2+\ldots + a_kn^k}{n^k} = 1$\\

Since $1$ is a constant, $F(n) \in O(n^k)$ and $F(n) \in \Omega(n^k)$. Thus,  $F(n) \in \Theta(n^k)$.\\

Furthermore, $\forall k\prime < k$, $\lim_{n\to\infty}  \frac{a_0+a_1n+a_2n^2+\ldots + a_kn^k}{n^k\prime}=\infty$.\\

Therefore, $F(n) \geq G(n) \rightarrow a_0+a_1n+a_2n^2+\ldots + a_kn^k \geq n^k\prime$.\\

Hence,  $a_0+a_1n+a_2n^2+\ldots + a_kn^k \in \Omega(n^{k\prime})$ but $\notin O(n^{k\prime})$ and thusly $\notin \Theta(n^{k\prime})$.\\

$$\blacksquare$$\\

\section{}

\textbf{Question:} Prove that $\log_2n = O(n^{1/3})$, but $\log_2n$ is not in $\Omega(n^{1/3})$. Is $\log_2n = \Theta(n^{1/3})$? Why or why not?\\

\subsection{Asymptotic Analysis}

By logarithmic properties and asymptotic analysis, $log_2(n) \in O(log(n))$. Thus,\\

$F(n)= log_2(n) = O(log(n))$\\

$G(n)=n^{1/3} = O(n^{1/3})$\\

$\lim_{n\to\infty} \frac{F(n)}{G(n)} = \frac{log(n)}{n^{1/3}} = 0$\\

Since the limit of $\frac{log_2(n)}{n^{1/3}}$ is $0$, then $\frac{log_2(n)}{n^{1/3}} < \infty$\\ and thus,\\

$$log_2(n) < n^{1/3}$$.\\

Hence, $F(n) \in O(G(n)) \rightarrow log_2(n) \in O(n^{1/3})$.\\

Thus, it is not possible for $log_2(n)$ to be $\Omega(n^{1/3})$ as $F(n) \ngeq G(n)$, and therefore $log_2(n) \notin \Theta(n^{1/3})$.\\

$$\blacksquare$$

\section{}

\textbf{Question:} Suppose the input array $A$ is in sorted order,
\emph{except} for $k$ elements. In other words, there are $n-k$ elements of $A$ that
are already in sorted order, and the remaining $k$ elements are out of order. Prove
that Insertion-Sort on $A$ runs in $O(nk)$ time.\\

\subsection{Pseudocode}

\begin{verbatim}
A = {A[0], A[1], ..., A[n]}
i = 1
swapper = 0

for i to n-1 {
    j = i
    while j > 0 && A[j-1] > A[j] {
        swapper = A[j-1]
        A[j-1] = A[j]
        A[j] = swapper                        
        j--
    }
}
\end{verbatim}

\subsection{Proof}
For insertion sort, after $k$ iterations the outer loop, the first $(k+1)$ elements in array $A$ are in sorted ascending order. Furthermore, the inner loop only iterates when $A[k-1] > A[k]$ and continues to iterate until $A=\{A[0], A[1], \ldots, A[k]\}$ is sorted in ascending order.\\

Therefore, the inner loop runs at most $k$ times, and the outer loop runs $n$ times. Using asymptotic analysis, $A \in O(nk)$.\\

$$\blacksquare$$



\end{document}
